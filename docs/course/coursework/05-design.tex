\chapter{Конструкторский раздел}

В данном разделе ...

\section{Требования к программному обеспечению}

Разрабатываемое программное обеспечение должно предоставлять пользователю следующую функциональность:

\begin{itemize}[label=--]
	\item изменение параметров алгоритма шума Перлина для генерации ландшафта;
	\item изменение положение источника света;
	\item возможность управления положением модели (перемещение, масштабирование, поворот);
	\item задание и изменение уровня моря в интерактивном режиме;
\end{itemize}

При этом разрабатываемая программа должна удовлетворять следующим требованиям:

\begin{itemize}[label=--]
	\item время отклика программы должно быть менее 1 секунды для корректной работы в интерактивном режиме;
	\item программа должна корректно реагировать на любые действия пользователя.
\end{itemize}

\section{Разработка алгоритмов}


\subsection{Алгоритм процедурной генерации ландшафта на основе шума Перлина}

На рисунке \ref{img:genHeightMapAlg} представлена схема алгоритма генерации карты высот на основе шума Перлина.

\includesvgimage
{genHeightMapAlg} % Имя файла без расширения (файл должен быть расположен в директории inc/img/)
{f} % Обтекание (без обтекания)
{h} % Положение рисунка (см. figure из пакета float)
{1\textwidth} % Ширина рисунка
{Схема алгоритма генерации карты высот на основе шума Перлина} % Подпись рисунка

\clearpage

На рисунке \ref{img:perlinNoiseWithMultiOctavesAlg} представлена схема алгоритма шума Перлина с применением нескольких октав.

\includesvgimage
{perlinNoiseWithMultiOctavesAlg} % Имя файла без расширения (файл должен быть расположен в директории inc/img/)
{f} % Обтекание (без обтекания)
{h} % Положение рисунка (см. figure из пакета float)
{0.65\textwidth} % Ширина рисунка
{Схема алгоритма шума Перлина с применением нескольких октав} % Подпись рисунка

\clearpage

На рисунке \ref{img:perlinNoiseAlg} представлена схема алгоритма шума Перлина.

\includesvgimage
{perlinNoiseAlg} % Имя файла без расширения (файл должен быть расположен в директории inc/img/)
{f} % Обтекание (без обтекания)
{h} % Положение рисунка (см. figure из пакета float)
{1\textwidth} % Ширина рисунка
{Схема алгоритма шума Перлина} % Подпись рисунка

\subsection{Модель освещения Ламберта}

В основе модели Ламберта лежит закон Ламберта \cite{info_lightLambert}, который утверждает, что интенсивность света, отраженного от материала, пропорциональна косинусу угла между нормалью к поверхности и направлением света. 
Это означает, что поверхности, ориентированные более к световому источнику, будут ярче, а те, которые ориентированы более в сторону от источника света, будут менее освещены.
На рисунке \ref{img:design_lambert} показан пример взаимного расположения вектора нормали (вектор $N$) и вектора направления от точки к источнику света (вектор $L$).

\includeimage
{design_lambert} % Имя файла без расширения (файл должен быть расположен в директории inc/img/)
{f} % Обтекание (без обтекания)
{h} % Положение рисунка (см. figure из пакета float)
{0.5\textwidth} % Ширина рисунка
{Пример расположения векторов $N$ и $L$ в модели освещения Ламберта \cite{info_lightModels}} % Подпись рисунка

Пусть:
\begin{itemize}
	\item $\vec{L}$ --- единичный вектор от точки до источника;
	\item $\vec{N}$ --- единичный вектор нормали;
	\item $I$ --- результирующая интенсивность света в точке;
	\item $I_0$ --- интенсивность источника;
	\item $K_d$ --- коэффициент диффузного освещения.
\end{itemize}

Формула расчета интенсивности имеет следующий вид:

\begin{equation}
	\label{equ:lambert}
	I = I_0 \cdot K_d \cdot cos(\vec{L}, \vec{N}) \cdot I_d = I_0 \cdot K_d \cdot (\vec{L}, \vec{N})
\end{equation}


\subsection{Алгоритм, использующий Z-буфер}

На рисунке \ref{img:zbufferAlg} представлена схема алгоритма  z-буфера.

\includesvgimage
{zbufferAlg} % Имя файла без расширения (файл должен быть расположен в директории inc/img/)
{f} % Обтекание (без обтекания)
{h} % Положение рисунка (см. figure из пакета float)
{0.8\textwidth} % Ширина рисунка
{Схема алгоритма z-буфера} % Подпись рисунка

\clearpage

\subsection{Алгоритм закраски по Гуро}

На рисунке \ref{img:shadingByGuroAlg} представлена схема алгоритма закраски по Гуро.

\includesvgimage
{shadingByGuroAlg} % Имя файла без расширения (файл должен быть расположен в директории inc/img/)
{f} % Обтекание (без обтекания)
{h} % Положение рисунка (см. figure из пакета float)
{0.65\textwidth} % Ширина рисунка
{Схема алгоритма закраски по Гуро} % Подпись рисунка

\clearpage

\subsection{Алгоритм, использующий Z-буфер, объединенный с закраской по Гуро}

На рисунке \ref{img:zbufferWithShadingByGuroAlg} представлена схема алгоритма, использующего Z-буфер, объединенного с закраской по Гуро.

\includesvgimage
{zbufferWithShadingByGuroAlg} % Имя файла без расширения (файл должен быть расположен в директории inc/img/)
{f} % Обтекание (без обтекания)
{h} % Положение рисунка (см. figure из пакета float)
{0.6\textwidth} % Ширина рисунка
{Схема алгоритма, использующего Z-буфер, объединенного с закраской по Гуро} % Подпись рисунка