\chapter{Конструкторский раздел}

%В данном разделе ...

%\section{Требования к программному обеспечению}
%
%Программное обеспечение должно удовлетворять следующим условиям:
%
%\begin{itemize}[label=--]
%	\item пользователь должен иметь возможность в реальном времени изменять параметры алгоритма шума Перлина для генерации ландшафта;
%	\item пользователь должен иметь возможность в реальном времени изменять положение источника света;
%	\item пользователь должен иметь возможность управлять положением модели (перемещение, масштабирование, поворот);
%	\item пользователь должен иметь возможность в реальном времени указывать уровень моя и изменять его в интерактивном режиме;
%	\item программа должна корректно реагировать на любые действия пользователя.
%\end{itemize}

%\section{Описание структур данных}
%
%Бла, бла, бла ...

\section{Разработка алгоритмов}

В данном подразделе будут приведены схемы работы алгоритмов

\subsection{Алгоритм Шума Перлина}

На рисунке \ref{img:genHeightMapAlg} представлена схема алгоритма генерации карты высот.

\includesvgimage
{genHeightMapAlg} % Имя файла без расширения (файл должен быть расположен в директории inc/img/)
{f} % Обтекание (без обтекания)
{h} % Положение рисунка (см. figure из пакета float)
{1\textwidth} % Ширина рисунка
{Схема алгоритма генерации карты высот} % Подпись рисунка

\clearpage

На рисунке \ref{img:perlinNoiseWithMultiOctavesAlg} представлена схема алгоритма шума Перлина с применением нескольких октав.

\includesvgimage
{perlinNoiseWithMultiOctavesAlg} % Имя файла без расширения (файл должен быть расположен в директории inc/img/)
{f} % Обтекание (без обтекания)
{h} % Положение рисунка (см. figure из пакета float)
{0.6\textwidth} % Ширина рисунка
{Схема алгоритма шума Перлина с применением нескольких октав} % Подпись рисунка

\clearpage

На рисунке \ref{img:perlinNoiseAlg} представлена схема алгоритма шума Перлина.

\includesvgimage
{perlinNoiseAlg} % Имя файла без расширения (файл должен быть расположен в директории inc/img/)
{f} % Обтекание (без обтекания)
{h} % Положение рисунка (см. figure из пакета float)
{0.8\textwidth} % Ширина рисунка
{Схема алгоритма шума Перлина} % Подпись рисунка

