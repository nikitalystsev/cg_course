\chapter{Технологический раздел}

В данном разделе приведены средства реализации ПО, листинги реализованных алгоритмов, описание интерфейса и примеры работы программы.

\section{Средства реализации}

Для реализации программного продукта был выбран язык программирования $C++$ \cite{info_pl} по следующим причинам:

\begin{itemize}[label*=---]
	\item данный язык преподавался в рамках курса по ООП;
	\item $C++$ обладает высокой вычислительной производительностью; \cite{info_cmpCplplPy, info_cmpCplplJava}, что очень важно для решения поставленной задачи;
	\item В стандартной библиотеке языка присутствует поддержка всех
	структур данных, выбранных по результатам проектирования;
	\item cредствами $C++$ можно реализовать все алгоритмы, выбранные в результате проектирования.
\end{itemize}

В качестве среды разработки предпочтение было отдано среде $QTCreator$ \cite{info_QtCr} по следующим причинам:

\begin{itemize}[label*=---]
	\item он обладает всем необходимым функционалом для написания, профилирования и
	отладки программ, а также создания графического пользовательского интерфейса;
	\item данная среда поставляется с фреймворком Qt \cite{info_QtDoc}, который содержит в себе все необходимые средства, позволяющие работать непосредственно с
	пикселями изображения;
	\item $QTCreator$ позволяет работать с расширением $QT Design$ \cite{info_QtDes}, который позволяет создавать удобный и надежный интерфейс.
\end{itemize}

Для упрощения и автоматизации сборки проекта используется утилита $cmake$ \cite{info_Cmake}.

\section{Реализация алгоритмов}

На листингах \ref{lst:genMultyOctavesNoise.txt} и \ref{lst:genCurrOctavesNoise.txt} представлена реализация алгоритма генерации шума Перлина.

На листинге \ref{lst:calcIntensityAtVertex.txt} представлена реализация функции расчета интенсивности света.

На листинге \ref{lst:transforms.txt} представлена реализация афинных преобразований поворота, масштабирования и переноса.

\includelisting
{genMultyOctavesNoise.txt} % Имя файла с расширением (файл должен быть расположен в директории inc/lst/)
{Реализация алгоритма шума Перлина с применением нескольких октав} % Подпись листинга

\clearpage

\includelisting
{genCurrOctavesNoise.txt} % Имя файла с расширением (файл должен быть расположен в директории inc/lst/)
{Реализация алгоритма шума Перлина для текущей октавы} % Подпись листинга

\includelisting
{calcIntensityAtVertex.txt} % Имя файла с расширением (файл должен быть расположен в директории inc/lst/)
{Реализация функции расчета интенсивности света согласно с моделью освещения Ламберта} % Подпись листинга

\clearpage

\includelisting
{transforms.txt} % Имя файла с расширением (файл должен быть расположен в директории inc/lst/)
{Реализация афинных преобразований} % Подпись листинга

\clearpage

\includelisting
{removeInvisiblePart1.txt} % Имя файла с расширением (файл должен быть расположен в директории inc/lst/)
{Реализация алгоритма закраски по Гуро в сочетании с алгоритмом удаления невидимых линий, использующим Z-буфер (часть 1)} % Подпись листинга

\includelisting
{removeInvisiblePart2.txt} % Имя файла с расширением (файл должен быть расположен в директории inc/lst/)
{Реализация алгоритма закраски по Гуро в сочетании с алгоритмом удаления невидимых линий, использующим Z-буфер (часть 2)} % Подпись листинга

\section{Интерфейс программы}


На рисунке \ref{img:interface1} интерфейс реализованного ПО.

\includeimage
{interface1} % Имя файла без расширения (файл должен быть расположен в директории inc/img/)
{f} % Обтекание (без обтекания)
{h} % Положение рисунка (см. figure из пакета float)
{1\textwidth} % Ширина рисунка
{интерфейс реализованного ПО} % Подпись рисунка

Пользователь может изменять вид ландшафта следующим образом:
\begin{enumerate}[label={\arabic*)}]
	 \item изменить параметры генерации;
	 \item изменить размеры ландшафта;
	 \item изменить положение и параметры точечного источника света;
	 \item изменить уровень моря c помощью слайдера или через поле ввода;
	 \item осуществить поворот, перенос и масштабирование ландшафта.
\end{enumerate}


\section{Демонстрация работы программы}

На рисунках \ref{img:ex1} показаны примеры работы программы.

\includeimage
{ex1} % Имя файла без расширения (файл должен быть расположен в директории inc/img/)
{f} % Обтекание (без обтекания)
{h} % Положение рисунка (см. figure из пакета float)
{1\textwidth} % Ширина рисунка
{Пример ландшафта} % Подпись рисунка

\includeimage
{ex2} % Имя файла без расширения (файл должен быть расположен в директории inc/img/)
{f} % Обтекание (без обтекания)
{h} % Положение рисунка (см. figure из пакета float)
{1\textwidth} % Ширина рисунка
{Пример изменения уровня моря в ландшафте} % Подпись рисунка

\includeimage
{ex3} % Имя файла без расширения (файл должен быть расположен в директории inc/img/)
{f} % Обтекание (без обтекания)
{h} % Положение рисунка (см. figure из пакета float)
{1\textwidth} % Ширина рисунка
{Пример применения операции поворота к ландшафту} % Подпись рисунка

