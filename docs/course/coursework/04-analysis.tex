\chapter{Аналитический раздел}

В данном разделе дано формальное описание объектов визуализируемой сцены, проанализированы разные способы представления данных о ландшафте, рассмотрены алгоритмы процедурной генерации ландшафта, алгоритмы удаления невидимых линий и поверхностей, алгоритмы закраски и модели освещения.

\section{Формализация объектов синтезируемой сцены}

Сцена состоит из следующих объектов:

\begin{itemize}[label=--]
	\item Ландшафт -- трехмерная модель, описываемая полигональной сеткой \cite{lands};
	\item Источник света -- материальная точка, испускающая лучи света.
\end{itemize}

\section{Анализ способов представления данных о ландшафте}

Существует несколько основных принципов представления данных для хранения информации о ландшафте \cite{datalands}:

\begin{itemize}[label=--]
	\item Регулярная сетка высот (карта высот);
	\item Иррегулярная сетка вершин и связей, их соединяющих;
	\item Посегментная карта высот.
\end{itemize}

\subsection{Регулярная сетка}

Данные представлены в виде двухмерного массива. Каждый элемент массива имеет свои индексы $[i, j]$, являющиеся координатами расположения точки на плоскости.  Для каждой вершины с индексами $[i, j]$ в двумерном массиве определяется соответствующее значение высоты $h_{ij}$.

\clearpage

\includeimage
{anal_heightmap} % Имя файла без расширения (файл должен быть расположен в директории inc/img/)
{f} % Обтекание (без обтекания)
{h} % Положение рисунка (см. figure из пакета float)
{0.8\textwidth} % Ширина рисунка
{Пример представления ландшафта с помощью карты высот} % Подпись рисунка

К преимуществам данного способа можно отнести наглядность и простоту изменения данных, легкость нахождения координат и высоты на карте, возможность более точно производить динамическое освещение. 

Однако, у такого метода есть один существенный недостаток -- избыточность данных (например, при задании плоскости).

\subsection{Иррегулярная сетка}

Этот метод представления данных о  ландшафте не использует равномерно распределенные узлы или точки, как в случае регулярной сетки. Вместо этого, иррегулярная сетка позволяет размещать точки или узлы в произвольных местах в зависимости от необходимости и особенностей ландшафта.

\includeimage
{anal_irregularmap} % Имя файла без расширения (файл должен быть расположен в директории inc/img/)
{f} % Обтекание (без обтекания)
{h} % Положение рисунка (см. figure из пакета float)
{0.3\textwidth} % Ширина рисунка
{Пример представления ландшафта с помощью иррегулярной сетки высот} % Подпись рисунка

У такого метода есть несколько существенных недостатков:

\begin{itemize}[label=--]
	\item многие алгоритмы построения ландшафтом предназначены для регулярных сеток высот, поэтому оптимизация этих алгоритмов под иррегулярную сетку потребует значительных усилий и времени;
	\item из-за неравномерного расположения вершинных точек друг к другу возникает сложность при создании динамического освещения;
	\item сложности при хранении, модификации и просмотре такого ландшафта.
\end{itemize}

К плюсам данного способа можно отнести использование меньшей информации для построения ландшафта \cite{datalands}. 

\subsection{Посегментная карта высот}

Суть этого метода заключается в разделении изначальной области на
небольшие участки (например, квадратной формы), и генерации высот на
каждом участке отдельно \cite{datalandsseg}.

К преимуществам данного способа можно отнести возможность представления больших открытых пространств, возможность хранения информации о других объектах (строения, пещеры, скалы), возможность создания нескольких вариантов одного и того же сегмента, но с разным уровнем детализаци.

К недостаткам можно отнести проблему стыковки разных сегментов, неочевидность представления данных, проблему модификации этих данных.

\section{Анализ алгоритмов процедурной генерации ландшафта}

В данном разделе рассмотрены различные алгоритмы процедурной генерации ландшафта, выделены преимущества и недостатки каждого метода. Основным критерием выбора алгоритма будет качество получаемого ландшафта, поскольку для решения задачи необходимо создавать правдоподобный рельеф.

\subsection{Алгоритм Diamond-Square}

Данный алгоритм является расширением одномерного алгоритма $midpoint \hspace{0.25cm} displacement$ \cite{midpdipl} на двумерную плоскость. Алгоритм $Diamond-Square$ \cite{diasqu} выполняется рекурсивно и начинает работу с двумерного массива размера $2^n + 1$. В четырёх угловых точках массива устанавливаются начальные значения высот. Шаги $diamond$ и $square$ выполняются поочередно до тех пор, пока все значения массива не будут установлены.

\begin{enumerate}[label={\arabic*)}]
    \item Шаг $diamond$ -- для каждого квадрата в массиве, устанавливается срединная точка, которой присваивается среднее арифметическое из четырёх угловых точек плюс случайное значение.
	\item Шаг $square$ -- берутся средние точки граней тех же квадратов, в которые устанавливается среднее значение от четырёх соседних с ними по осям точек плюс случайное значение.
\end{enumerate}

\includeimage
{anal_dia_squ} % Имя файла без расширения (файл должен быть расположен в директории inc/img/)
{f} % Обтекание (без обтекания)
{h} % Положение рисунка (см. figure из пакета float)
{1\textwidth} % Ширина рисунка
{Шаги, проходимые алгоритмом Diamond-Square на примере массива 5х5} % Подпись рисунка

К преимуществам данного алгоритма можно отнести простоту его реализации и возможность генерации различных ландшафтов с разнообразной детализацией.

К недостаткам можно отнести создание вертикальных и горизонтальных <<складок>> на краях карты из-за наиболее значительного возмущения, происходящего в прямоугольной сетке \cite{diasquwiki}, а также сложности с контролем получаемого ландшафта.

\subsection{Холмовой алгоритм}

Это простой итерационный алгоритм, основанный на нескольких входных параметрах. Алгоритм изложен в следующих шагах:

\begin{enumerate}[label={\arabic*)}]
	\item Создается двухмерный массив и инициализируется нулевым уровнем (заполнются все ячейки нолями);
	\item Берется случайная точку на ландшафте или около его границ (за границами) и случайный радиус в заранее заданных пределах. Выбор этих пределов влияет на вид ландшафта -- либо он будет пологим, либо скалистым; 
	\item В выбранной точке <<поднимается>> холм заданного радиуса;
	\item Выполнение пунктов 1) - 3) $n$ раз, где $n$ -- выбранное количество шагов. От него потом будет зависеть внешний вид нашего ландшафта;
	\item Проведение нормализации ландшафта;
	\item Проведение <<долинизации>> ландшафта.
\end{enumerate}

Холм – половина шара, похоже на перевернутую параболу. Выбранный радиус определяет высоту холма и радиус его основания. Уравнение холма выглядит так: 

\begin{equation}
	\label{equ:hill}
	z = r^2 - ((x_2 - x_1)^2 + (y_2 - y_1)^2)
\end{equation}

Где $(x_1, y_1)$ -- заданная точка, $r$ -- выбранный радиус, $(x_2, y_2)$ -- высота холма.

Для генерации правдоподобного ландшафта необходимо построить множество холмов. При генерации необходимо игнорировать отрицательные значения высоты холма, а также при генерации последующих холмов лучше добавлять полученное значение для данного холма к уже существующим значениям \cite{datalands}. 

Для модификации и масштабирования ландшафта необходимо произвести процесс нормализации ландшафта. Для добавления долин в полученный сгенерированный ландшафт, применяют процесс долинизации ландшафта \cite{hillalg}.

К преимуществам данного алгоритма можно отнести простоту его реализации, а также возможность контроля <<гористости>> ландшафта за счет этапов нормализации и долинизации.

Данный алгоритм имеет несколько недостатков:

\begin{itemize}[label=--]
	\item c помощью этого алгоритма тяжело смоделировать склон холма, или
	горы, так как в процессе генерации ландшафта используются только
	гладкие полусферы \cite{hillalgflaws};
	\item этот алгоритм неприменим, когда требуется детализировано
	отобразить лишь часть всего ландшафта \cite{hillalgflaws}.
\end{itemize}

\subsection{Алгоритм Шума Перлина}

Это математический алгоритм по генерированию процедурной
текстуры псевдо-случайным методом \cite{perlnoise}. Этот алгоритм может быть реализован для n-мерного пространства, но чаще его используют для одно-, двух-, трехмерного случая.

Рассмотрим версию этого алгоритма для двумерного случая:

Для карты высот создается сетка точек и в каждой точке сетки генерируется псевдослучайный единичный вектор-направления градиента. Для каждой точки с координатами $(x, y)$ из карты высот определяется, в какой ячейке сетки находится точка, генерируются вектора, идущие от точек ячейки сетки до этой точки.

\includeimage
{perlin} % Имя файла без расширения (файл должен быть расположен в директории inc/img/)
{f} % Обтекание (без обтекания)
{h} % Положение рисунка (см. figure из пакета float)
{0.35\textwidth} % Ширина рисунка
{Пример сгенерированных векторов для точки $(x, y)$} % Подпись рисунка

Вычисляются четыре скалярных произведения векторов-направлений
градиента и вектора, идущего от связанной с ним точки сетки к точке с
координатами $(x, y)$.

Далее, чтобы получить значение высоты $z$ в точке $(x, y)$, необходимо провести двумерную интерполяцию на основе полученных скалярных произведений. Для создания плавного перехода между значениями предварительно производятся вычисления весов измерений по $x$ и по $y$. Веса определяются функцией

\begin{equation}
	\label{equ:perlin}
	smootherstep(t) = 6t^5 - 15t^4 + 10t^3
\end{equation}
	
где вместо $t$ подставляются значения $x$ и $y$. Далее, используя метод последовательной интерполяции по каждому измерению, получаем значения двух одномерных линейных интерполяций с использованием веса по $x$, и в конце проводим одну одномерную линейную интерполяцию с этими вычисленными значениями, но уже по весу измерения $y$. Полученное значение и будет высотой в данной точке
$(x, y)$ карты высот. 

Чтобы контролировать генерацию шума, существует набор параметров:

\begin{itemize}[label=--]
	\item Октавы -- количество уровней детализации шума;
	\item Лакунарность -- множитель, который определяет изменение частоты с ростом октавы;
	\item Стойкость -- множитель частотной амплитуды, который
	определяет изменение амплитуды с ростом октавы.
\end{itemize}

\clearpage

Для достижения качественного детализированного ландшафта можно смешивать шумы разных частот и амплитуд:

\includeimage
{perlin_octaves} % Имя файла без расширения (файл должен быть расположен в директории inc/img/)
{f} % Обтекание (без обтекания)
{h} % Положение рисунка (см. figure из пакета float)
{1\textwidth} % Ширина рисунка
{Изменение детализации шума Перлина с применением нескольких октав} % Подпись рисунка

Преимущества:

\begin{itemize}[label=--]
	\item комбинация различных октав в алгоритме шума Перлина позволяет
	добавлять дополнительные уровни детализации к сгенерированному
	шуму, тем самым повышая реалистичность и детализацию 
	ландшафта;
	\item зная лишь параметры генерации, можно получить высоту в любой
	точке карты без необходимости знания высот в соседних точках
	карты.
\end{itemize}

Недостатки:

\begin{itemize}[label=--]
	\item без использования механизма комбинации октав сгенерированный
	ландшафт не выглядит реалистичным и детализированным.
\end{itemize}

\section{Анализ алгоритмов удаления невидимых линий и поверхностей}

Задача удаления невидимых линий и поверхностей является одной из
наиболее сложных в машинной графике. Алгоритмы удаления невидимых линий
и поверхностей служат для определения линий ребер, поверхностей или
объемов, которые видимы или не видимы для наблюдателя, находящегося в 
данной точке пространства \cite{rodjers}.
 
Главным требованием при выборе алгоритма будут высокая скорость
работы, чтобы пользователю не приходилось ждать долгой загрузки
изображения.

\subsection{Алгоритм Робертса}

Алгоритм Робертса работает в объектном пространстве только с
выпуклыми телами. Если тело не является выпуклым, то его предварительно
нужно разбить на выпуклые составляющие \cite{rodjers}.

Этот алгоритм выполняется в 4 этапа:

\begin{enumerate}[label={\arabic*)}]
	\item Подготовка исходных данных – составление матрицы тела для
	каждого тела сцены;
	\item Удаление ребер, экранируемых самим телом;
	\item Удаление ребер, экранируемых другими телами;
	\item Удаление линий пересечения тел, экранируемых самими телами и
	другими телами, связанными отношением протыкания.
\end{enumerate}

Алгоритм работает только с выпуклыми телами, что является недостатком данного алгоритма. Также к недостатку можно отнести то, что вычислительная сложность теоретически растет как квадрат числа объектов. К преимуществам можно отнести высокую точность вычислений.

\subsection{Алгоритм Варнока}

Алгоритм Варнока работает в пространстве изображения. Экран
рассматривается как окно и решается вопрос о том, пусто ли оно или его
содержимое достаточно просто для визуализации. Если это не так, то окно
разбивается на подокна до тех пор, пока содержимое подокна не станет
достаточно простым для визуализации или его размер не достигнет требуемого 
предела разрешения. Если информации достаточно, то происходит ее усреднение
и результат отображается с одинаковой интенсивностью или цветом \cite{rodjers}.

Пределом разбиения для растрового экрана в случае простого алгоритма
является размер окна в 1 пиксель. Единой версии этого алгоритма не существует,
есть только его различные модификации.

К недостаткам данного алгоритма можно отнести увеличении значительное увеличение числа разбиений при увеличении сложности сцены, что может негативно сказаться на скорости работы алгоритма.

\subsection{Алгоритм, использующий Z-буфер}

Это один из простейших алгоритмов удаления невидимых поверхностей. Работает этот алгоритм в пространстве изображения\cite{rodjers}. 

В данном алгоритме используется два буфера: буфер кадра и z-буфер. Буфер кадра используется для заполнения атрибутов (интенсивности) каждого пикселя в пространстве изображения. z-буфер -- отдельный буфер глубины, используемый для запоминания координаты z или глубины каждого видимого пикселя в пространстве изображения \cite{rodjers}.

Вначале в z-буфер заносятся минимально возможные значения $z$, а буфер
кадра заполняется заполняются фоновым значением интенсивности или цвета.
Затем каждый многоугольник преобразовывается в растровую форму. Суть 
работы алгоритма заключается в следующем: в процессе работы глубина
(значение координаты $z$) каждого нового пикселя, который надо занести в буфер
кадра сравнивается с глубиной того пикселя, который уже есть в z-буфере. Если
новый пиксель оказывается расположен ближе к наблюдателю, то новый пиксель
заносится в буфер кадра. При этом в z-буфер заносится глубина нового пикселя.
Если сравнение дало противоположный результат, то никаких действий не
производится. 

К преимуществам данного алгоритма относится простота его реализации, возможность работы со сценами любой сложности, отсутствие необходимости в сортировке по приоритету глубины.

К недостаткам можно отнести большой объем требуемой памяти, трудоемкость и высокая стоимость устранения лестничного эффекта, трудоемкость реализации эффектов прозрачность и просвечивания.

\clearpage 

\subsection{Алгоритм обратной трассировки лучей}

Наблюдатель видит объект посредством испускаемого источником света,
который падает на этот объект и согласно законам оптики некоторым путем
доходит до глаза наблюдателя. Алгоритм имеет такое название, потому что
эффективнее с точки зрения вычислений отслеживать пути лучей в обратном
направлении, то есть от наблюдателя к объекту.

Преимущества:

\begin{itemize}[label=--]
	\item Возможность использования алгоритма в параллельных
	вычислитель системах;
	\item Высокая реалистичность получаемого изображения.
\end{itemize}

Недостатки:

\begin{itemize}[label=--]
	\item Большое количество вычислений и медленная работа алгоритма.
\end{itemize}

\section{Анализ моделей освещения}

Все модели освещения делятся на две группы: глобальные и локальные.

Алгоритмы глобальных моделей освещения учитывают не только свет, который поступает непосредственно от источника света (прямое освещение), но и последующие случаи, в которых световые лучи от того же источника отражаются другими поверхностями сцены, отражающими или нет (непрямое освещение) \cite{globill}.

Алгоритмы локальных моделей освещения учитывают только тот свет, который поступает непосредственно от источника света. Выделяют две основные модели локального освещения: модель Ламберта и модель Фонга.

\subsection{Модель освещения Ламберта}

Модель Ламберта моделирует идеальное диффузное освещение. Считается,
что свет при попадании на поверхность рассеивается равномерно во все стороны.
При расчете такого освещения учитывается только ориентация поверхности  (нормаль $N$) и направление на источник света (вектор $L$). Рассеянная составляющая рассчитывается по
закону косинусов (закон Ламберта) \cite{modli}: 

\clearpage

\includeimage
{lamb} % Имя файла без расширения (файл должен быть расположен в директории inc/img/)
{f} % Обтекание (без обтекания)
{h} % Положение рисунка (см. figure из пакета float)
{0.5\textwidth} % Ширина рисунка
{Пример расположения векторов $N$ и $L$ в модели освещения Ламберта} % Подпись рисунка

Для удобства все векторы, описанные ниже, берутся единичными. В этом случае косинус угла между ними совпадает со скалярным произведением. Формула расчета интенсивности имеет следующий вид:


\begin{equation}
	\label{equ:lambert}
	I = I_0 \cdot K_d \cdot cos(\vec{L}, \vec{N}) \cdot I_d = I_0 \cdot K_d \cdot (\vec{L}, \vec{N}) \cdot I_d
\end{equation}

Где $I$ -- результирующая интенсивность света в точке, $I_0$ -- интенсивность источника, $K_d$ -- коэффициент диффузного освещения, $\vec{L}$ -- вектор от точки до
источника, $\vec{N}$ -- вектор нормали в точке, $I_d$ -- мощность рассеянного освещения.

\subsection{Модель освещения Фонга}

Модель Фонга – классическая модель освещения. Модель представляет собой комбинацию диффузной составляющей (модели Ламберта) и зеркальной составляющей и работает таким образом, что кроме равномерного освещения на  материале может еще появляться блик. Местонахождение блика на объекте, освещенном по модели Фонга, определяется из закона равенства углов падения и отражения. Если наблюдатель находится вблизи углов отражения, яркость соответствующей точки повышается \cite{modli}.

\clearpage

\includeimage
{fongsum} % Имя файла без расширения (файл должен быть расположен в директории inc/img/)
{f} % Обтекание (без обтекания)
{h} % Положение рисунка (см. figure из пакета float)
{1\textwidth} % Ширина рисунка
{Составляющие модели Фонга (слева направо: фоновая, диффузная и зеркальная)} % Подпись рисунка

\includeimage
{fong} % Имя файла без расширения (файл должен быть расположен в директории inc/img/)
{f} % Обтекание (без обтекания)
{h} % Положение рисунка (см. figure из пакета float)
{0.5\textwidth} % Ширина рисунка
{Пример расположения векторов в модели освещения Фонга} % Подпись рисунка

Формула для расчета интенсивности для модели Фонга имеет вид:

\begin{equation}
	\label{equ:fong}
	I = K_a \cdot I_a + I_0 \cdot K_d \cdot (\vec{L}, \vec{N}) \cdot I_d + I_0 \cdot K_s \cdot (\vec{R}, \vec{V})^a \cdot I_s
\end{equation}

Где $I$ -- результирующая интенсивность света в точке, $K_a$ -- коэффициент фонового освещения; $L_a$ -- интенсивность фонового освещения, $I_0$ -- интенсивность источника, $K_d$ -- коэффициент диффузного освещения; $\vec{L}$ -- вектор от точки до источника; $\vec{N}$ -- вектор нормали в точке; $I_d$ -- интенсивность диффузного освещения; $K_s$ -- коэффициент зеркального освещения; $\vec{R}$ -- вектор отраженного луча; $\vec{V}$ -- вектор от точки до наблюдателя; $a$ -- коэффициент блеска; $I_s$ -- интенсивность зеркального освещения.

\section{Анализ алгоритмов закраски}

Методы закраски используются для затенения полигонов модели в
условиях некоторой сцены, имеющей источники освещения. Существует три
основных алгоритма, позволяющих закрасить полигональную модель.

\subsection{Алгоритм простой закраски}

Суть данного алгоритма заключается в том, что для каждой грани объекта
находится вектор нормали, и с его помощью в соответствии с выбранной
моделью освещения вычисляется значение интенсивности, с которой
закрашивается вся грань.
При данной закраске все плоскости (в том числе и те, что аппроксимируют
фигуры вращения), будут закрашены однотонно, что в случае с фигурами
вращения будет давать ложные ребра \cite{rodjers}.

К преимуществам можно отнести простоту реализации алгоритма и его быстродействие. В качестве недостатков можно выделить нереалистичность результата.

\subsection{Алгоритм закраски по Гуро}

Данный алгоритм позволяет получить более сглаженное изображение. Это
достигается благодаря тому, что разные точки грани закрашиваются разным
значением интенсивности.

Алгоритм состоит из следующих шагов:

\begin{enumerate}[label={\arabic*)}]
	\item Вычисление векторов нормалей к каждой грани;
	\item Вычисление векторов нормали к каждой вершине грани путем усреднения нормалей примыкающих граней;
	\item Вычисление интенсивности в вершинах грани в соответствии с выбранной моделью освещения;
	\item Выполнение линейной интерполяции интенсивности вдоль ребер;
	\item Выполнение линейной интерполяции вдоль сканирующей строки.
\end{enumerate}

Преимуществами данного алгоритма являются хорошее сочетание с моделью освещения с диффузным отражением, а также более высокая, по сравнению с алгоритмом простой закраски, реалистичность получаемого изображения.

К недостаткам можно отнести отсутствие учета кривизны поверхности \cite{rodjers}.

\subsection{Алгоритм закраски по Фонгу}

В алгоритме закраски по Фонгу используется билинейная интерполяция не интенсивностей в вершинах полигона, а билинейная интерполяция векторов нормалей. Благодаря такому подходу изображение получается более реалистичным. Однако для достижения такого результата требуется больше вычислительных затрат \cite{rodjers}.

Данный алгоритм состоит из следующих шагов:

\begin{enumerate}[label={\arabic*)}]
	\item Вычисление векторов нормалей к каждой грани;
	\item Вычисление векторов нормали к каждой вершине грани путем усреднения нормалей примыкающих граней;
	\item Выполнение линейной интерполяции нормалей вдоль ребер;
	\item Выполнение линейной интерполяции нормалей вдоль сканирующей строки.
	\item Вычисление интенсивности в каждой вершине
\end{enumerate}


\clearpage

\section*{Вывод}

% по точке выравнивание
\begin{table}[ht]
	\small
	\begin{center}
		\begin{threeparttable}
			\caption{Сравнение способов представления данных о ландшафте}
			\label{tbl:datalands}
			\begin{tabular}{|c|c|c|}
				\hline
				Способ & \makecell{Наглядность \\ представления \\ данных} & \makecell{Сложность \\ модификации \\ данных} \\
				\hline
				\makecell{Регулярная сетка} & Высокая & Низкая  \\
				\hline
				\makecell{Иррегулярная сетка} & Высокая & Средняя  \\
				\hline
				\makecell{Посегментная карта высот} & Средняя & Высокая  \\
				\hline
			\end{tabular}
		\end{threeparttable}			
	\end{center}
\end{table}     


% по точке выравнивание
\begin{table}[ht]
	\small
	\begin{center}
		\begin{threeparttable}
			\caption{Сравнение алгоритмов процедурной генерации ландшафта}
			\label{tbl:alggenlands}
			\begin{tabular}{|c|c|c|c|}
				\hline
				Алгоритм & \makecell{Качество \\ ландшафта} & \makecell{Отстутствие \\ артефактов} & \makecell{Контроль \\ ландшафта} \\
				\hline
				Diamond-Square & Среднее & -- & Низкая   \\
				\hline
				\makecell{Холмовой \\ алгоритм} & Среднее & + & Средний \\
				\hline
				Шум Перлина & Высокое & + & Высокая  \\
				\hline
			\end{tabular}
		\end{threeparttable}			
	\end{center}
\end{table}      

% по точке выравнивание
\begin{table}[ht]
	\small
	\begin{center}
		\begin{threeparttable}
			\caption{Сравнение алгоритмов удаления невидимых линий и поверхностей}
			\label{tbl:algdel}
			\begin{tabular}{|c|c|c|c|}
				\hline
				Алгоритм & \makecell{Сложность \\ алгоритма} & \makecell{Скорость \\ работы} & \makecell{Типы \\ объектов} \\
				\hline
				\makecell{Алгоритм Робертса} & $O(n^2)$ & Средняя & \makecell{Выпуклые \\ многогранники}   \\
				\hline
				\makecell{Алгоритм Варнока} & $O(np)$ & Средняя & Произвольные \\
				\hline
				\makecell{Алгоритм с $z$-буфером} & $O(np)$ & Высокая & Произвольные  \\
				\hline
				\makecell{Алгоритм с обратной \\ трассировки лучей} & $O(np)$ & Низкая & Произвольные  \\
				\hline
			\end{tabular}
		\end{threeparttable}			
	\end{center}
\end{table}   

\clearpage 

% по точке выравнивание
\begin{table}[ht]
	\small
	\begin{center}
		\begin{threeparttable}
			\caption{Сравнение моделей освещения}
			\label{tbl:modellight}
			\begin{tabular}{|c|c|c|}
				\hline
				Модель освещения & \makecell{Реалистичность \\ изображения} & \makecell{Объем вычислений} \\
				\hline
				\makecell{Модель Ламберта} & Средняя & Средний   \\
				\hline
				\makecell{Модель Фонга} & Высокая & Большой \\
				\hline
			\end{tabular}
		\end{threeparttable}			
	\end{center}
\end{table} 

% по точке выравнивание
\begin{table}[ht]
	\small
	\begin{center}
		\begin{threeparttable}
			\caption{Сравнение алгоритмов закраски}
			\label{tbl:draw}
			\begin{tabular}{|c|c|c|c|}
				\hline
				Алгоритм закраски & \makecell{Скорость \\ работы} & \makecell{Реалистичность \\ изображения} & \makecell{Сочетание с \\ диффузным отражением} \\
				\hline
				\makecell{Простая закраска} & Высокая & Низкая & Высокое  \\
				\hline
				\makecell{Закраска по Гуро} & Средняя & Средняя & Высокое \\
				\hline
				\makecell{Закраска по Фонгу} & Низкая & Высокая & Cредняя \\
				\hline
			\end{tabular}
		\end{threeparttable}			
	\end{center}
\end{table} 
  
В данном разделе были рассмотрены существующие методы для визуализации и построения трехмерного ландшафта. В качестве способа представления данных о ландшафте была выбрана регулярная карта высот ввиду своей простоты представления данных, в качестве алгоритма генерации карты высот был выбран алгоритм шума Перлина. Поскольку основным критерием  выбора алгоритмов была скорость их работы, то для удаления невидимых линий и поверхностей был выбран алгоритм, использующий Z-буффер, а в качестве модели освещения и алгоритма закраски были выбраны модель Ламберта и алгоритм закраски по Гуро. 

                                                                                                                                                                                                     