\chapter{Аналитический раздел}

В данном разделе будет дано описание алгоритмов визализации ... (дописать)

\section{Формализация объектов синтезируемой сцены}

Сцена состоит из следующих объектов:

\begin{itemize}[label=--]
	\item Ландшафт -- трехмерная модель, представляющая собой полигональную сетку, состоящую из связанных между собой плоских многоугольников;
	\item Источник света -- материальная точка пространства, испускающая лучи света.
\end{itemize}

\section{Анализ и выбор формы задания трехмерных моделей}

В компьютерной графике для описания трехмерных объектов существует три вида моделей: каркасная, поверхностная и твердотельная. Использование моделей позволяет правильно отображать форму и размеры объектов сцены.

\begin{itemize}[label=--]
	\item Каркасная модель -- в трехмерной графике описывает совокупность вершин и ребер, которая показывает форму многогранного объекта. Это моделирование самого низкого уровня и имеет ряд серьезных ограничений, большинство из которых возникает из-за недостатка информации о гранях, которые заключены между линиями, и невозможности выделить внутреннюю и внешнюю область изображения твердого объемного тела. Однако каркасная модель требует меньше памяти и вполне пригодна для решения задач, относящихся к простым. Основным недостатком каркасной модели является то, что модель не всегда однозначно передает информацию о форме объекта; 
	\item Поверхностная модель. Поверхностное моделирование определяется в терминах точек, линий и поверхностей. При построении поверхностной модели предполагается, что технические объекты ограничены поверхностями, которые отделяют их от окружающей среды. Недостатком поверхностной модели является отсутствие информации о том, с какой стороны находится поверхности материал;
	\item Твердотельная модель. Отличие данной формы задания модели от поверхностной формы состоит в том, что в объёмных моделях к информации о поверхностях добавляется информация о том, где расположен материал путём указания направления внутренней нормали.
\end{itemize}

Для решения моей задачи использование каркасной модели не подойдет, поскольку та дает не однозначного представления о форме ландшафта, также использование твердотельной модели не дает каких-либо преимуществ, поскольку для моей задачи нет необходимости знать физические свойства ландшафта. Поэтому наилучшим способом описания трехмерной модели ландшафта будет поверхностная модель.


\section{Анализ способов представления данных о ландшафте}

Существует несколько основных принципов представления данных для хранения информации о ландшафтах:

\begin{itemize}[label=--]
	\item Использование регулярной сетки высот (карта высот);
	\item Использование иррегулярной сетки вершин и связей, их соединяющих (хранение простой триангулизированной карты);
	\item Хранение карты ландшафта, но в данном случае хранятся не конкретные высоты, а информация об использованном блоке. В этом случае создается некоторое количество заранее построенных сегментов, а на карте указываются только индексы этих сегментов.
\end{itemize}

\subsection{Карта высот}

Данные представлены в виде двухмерного массива. Для каждой вершины с индексами $[i][j]$ в двумерном массиве определяется соответствующее значение высоты $h_{ij}$.

\clearpage

\includeimage
{anal_heightmap} % Имя файла без расширения (файл должен быть расположен в директории inc/img/)
{f} % Обтекание (без обтекания)
{h} % Положение рисунка (см. figure из пакета float)
{0.8\textwidth} % Ширина рисунка
{Представление ландшафта с помощью регулярной сетки высот} % Подпись рисунка

Преимущества:

\begin{itemize}[label=--]
	\item наглядность, простота изменения данных;
	\item легкость нахождения координат и высоты на карте;
	\item из-за близкого друг к другу расположения вершинных точек можно 
	более точно производить динамическое освещение.
\end{itemize}

Недостатки:

\begin{itemize}[label=--]
	\item слишком много описаний для точек;
	\item избыточность данных, например, при задании плоскости.
\end{itemize}

\subsection{Иррегулярная сетка}

Это метод представления ландшафта, который не использует равномерно распределенные узлы или точки, как в случае регулярной сетки. Вместо этого, иррегулярная сетка позволяет размещать точки или узлы в произвольных местах в зависимости от необходимости и особенностей ландшафта.

\clearpage

\includeimage
{anal_irregularmap} % Имя файла без расширения (файл должен быть расположен в директории inc/img/)
{f} % Обтекание (без обтекания)
{h} % Положение рисунка (см. figure из пакета float)
{0.4\textwidth} % Ширина рисунка
{Представление ландшафта с помощью иррегулярной сетки высот} % Подпись рисунка

Преимущества:

\begin{itemize}[label=--]
	\item используется меньше информации для построения ландшафта. Необходимо хранить только значения высот каждой вершины и связи, эти вершины соединяющие.
\end{itemize}

Недостатки:

\begin{itemize}[label=--]
	\item многие алгоритмы построения ландшафтом предназначены для регулярных сеток высот, поэтому оптимизация этих алгоритмов под иррегулярную сетку потребует значительных усилий и времени;
	\item из-за неравномерного расположения вершинных точек друг к другу возникает сложность при создании динамического освещения;
	\item сложности при хранении, модификации и просмотре такого ландшафта.
\end{itemize}

\subsection{Посегментная карта высот}

В данном способе также используются карты высот. Только вместо высот в ней хранятся индексы ландшафтных сегментов. Как эти сегменты представлены, в принципе, роли не играет. Они могут быть и регулярными, и иррегулярными (причем можно использовать и те и другие одновременно).

\clearpage

Преимущества: 

\begin{itemize}[label=--]
	\item возможность представления больших открытых пространств;
	\item кроме самих ландшафтов в таких блоках можно хранить и информацию о зданиях, строениях, растениях, специфических ландшафтных решениях (например, пещеры или скалы, нависающие друг над другом);
	\item возможность создания нескольких вариантов одного и того же сегмента, но при разной степени детализации. В зависимости от скорости или загруженности компьютера можно выбирать более или менее детализованные варианты.
\end{itemize}

Недостатки:

\begin{itemize}[label=--]
	\item проблема стыковки разных сегментов;
	\item неочевидность данных. Взглянув на картинку, вы не сможете моментально представить, как это должно будет выглядеть в игре;
	\item проблема модификации.
	Проанализировав все три способа представления данных о ландшафте, выбор пал на регулярную сетку высот, поскольку: 
	\item Первый способ дает более наглядное и понятное представление данных, позволяет легко их модифицировать. Также для этого способа существует множество алгоритмов построения ландшафта;
	\item При использовании иррегулярной сетки возникают сложности с хранением и модификацией данных. Также оптимизация существующих алгоритмов построения ландшафта под использование этого способа отнимет немало времени и усилий;
	\item Третий способ также влечет за собой проблемы с хранением и модификацией данных.
\end{itemize}

\clearpage

\section{Анализ алгоритмов процедурной генерации ландшафта}

В данном разделе рассмотрены различные методы и алгоритмы процедурной генерации ландшафта. Также рассмотрены преимущества и недостатки каждого метода. Основным критерием выбора алгоритма будет качество получаемого ландшафта, поскольку для решения задачи необходимо создавать правдоподобный рельеф.

\subsection{Алгоритм Diamond-Square}

Данный алгоритм является расширением одномерного алгоритма $midpoint \hspace{0.25cm} displacement$ на двумерную плоскость. Алгоритм $Diamond-Square$ начинает работу с двумерного массива размера $2^n + 1$. В четырёх угловых точках массива устанавливаются начальные значения высот. Шаги $diamond$ и $square$ выполняются поочередно до тех пор, пока все значения массива не будут установлены.

\begin{enumerate}[label={\arabic*)}]
    \item Шаг $diamond$. Для каждого квадрата в массиве, устанавливается срединная точка, которой присваивается среднее арифметическое из четырёх угловых точек плюс случайное значение.
	\item Шаг $square$. Берутся средние точки граней тех же квадратов, в которые устанавливается среднее значение от четырёх соседних с ними по осям точек плюс случайное значение.
\end{enumerate}

\includeimage
{anal_dia_squ} % Имя файла без расширения (файл должен быть расположен в директории inc/img/)
{f} % Обтекание (без обтекания)
{h} % Положение рисунка (см. figure из пакета float)
{1\textwidth} % Ширина рисунка
{Шаги, проходимые алгоритмом Diamond-Square на примере массива 5х5} % Подпись рисунка

\clearpage

Преимущества: 

\begin{itemize}[label=--]
	\item алгоритм Diamond-Square достаточно прост в реализации, особенно по сравнению с некоторыми другими алгоритмами процедурной генерации ландшафта. Он может быть реализован относительно легко и быстро на различных платформах;
	\item алгоритм Diamond-Square позволяет быстро генерировать реалистичные и природные ландшафты с разнообразной текстурой и детализацией. 
\end{itemize}

Недостатки:

\begin{itemize}[label=--]
	\item алгоритм создает заметные вертикальные и горизонтальные "складки" на краях карты из-за наиболее значительного возмущения, происходящего в прямоугольной сетке;
	\item при использовании этого алгоритма возникают сложность с контролем получаемого ландшафта. 
\end{itemize}

\subsection{Холмовой алгоритм}

Это простой итерационный алгоритм, основанный на нескольких входных параметрах. Алгоритм изложен в следующих шагах:

\begin{enumerate}[label={\arabic*)}]
	\item создаем двухмерный массив и инициализируем его нулевым уровнем (заполняем все ячейки нолями);
	\item берем случайную точку на ландшафте или около его границ (за границами), а также берем случайный радиус в заранее заданных пределах. Выбор этих пределов влияет на вид ландшафта -- либо он будет пологим, либо скалистым; 
	\item В выбранной точке "поднимаем" холм заданного радиуса;
	\item Возвращаемся ко второму шагу и так далее до выбранного количества шагов. От него потом будет зависеть внешний вид нашего ландшафта;
	\item Проводим нормализацию ландшафта;
	\item Проводим "долинизацию" ландшафта. Делаем его склоны более пологими.
\end{enumerate}

Холм – половина шара, похоже на перевернутую параболу. Выбранный радиус определяет высоту холма и радиус его основания. Уравнение холма выглядит так: 
