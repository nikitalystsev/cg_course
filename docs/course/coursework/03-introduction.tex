
\chapter*{ВВЕДЕНИЕ}
\addcontentsline{toc}{chapter}{ВВЕДЕНИЕ}

% проверено мной: пойдет

В настоящее время технологии трехмерной графики стремительно развиваются.
Одним из направлений применения этих технологий является создание видеоигр.
Наибольшую популярность набирают игры с так называемым открытым миром.

Основой для открытого мира служит трехмерный ландшафт значительных размеров, который для реалистичности должен быть разнообразным и детализированным.
Традиционные методы ручного моделирования ландшафта являются крайне громоздкими, трудоемкими и ограниченными в своей вариативности. 

Возникает потребность в создании программного обеспечения, которое бы позволяло автоматизировать процессы создания реалистичного ландшафта, чтобы ускорить разработку игр и обеспечить большую степень креативной свободы для разработчиков.

Целью работы является разработка программного обеспечения для генерации и визуализации трехмерного ландшафта.


Для достижения поставленной цели необходимо решить следующие задачи: 

\begin{enumerate}[label={\arabic*)}]
	\item выполнить формализацию объектов синтезируемой сцены;
	\item провести анализ существующих алгоритмов создания ландшафта и визуализации сцены, выбрать подходящие и обосновать их выбор;
	\item разработать ПО;
	\item реализовать выбранные алгоритмы;
	\item провести исследование временных характеристик разработанного ПО.
\end{enumerate}

